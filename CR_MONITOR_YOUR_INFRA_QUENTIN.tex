\documentclass[10pt,a4paper]{article}
\usepackage[utf8]{inputenc}
\usepackage[french]{babel}
\usepackage[left=2cm,right=2cm,top=2cm,bottom=2cm]{geometry}
\usepackage{hyperref}
\usepackage{graphicx}

% \usepackage{fancyhdr}
% \fancyhead{}
% \fancyfoot{}
% \fancyhead[L]{\includegraphics[scale=0.03]{../image/logo_iutbeziers.png}}
% \fancyhead[C]{Rapport de Stage - SuperBeeLive}
% 
% \fancyfoot[L]{\small Olivia SERENELLI-PESIN \normalsize}
% \fancyfoot[R]{\thepage/\pageref{LastPage}}
% %\fancyfoot[C]{\includegraphics[scale=0.03]{../images/logo/logo_abeille.png}}
% \renewcommand{\footrulewidth}{0pt}
% \renewcommand{\headrulewidth}{0,4pt}
% 
% %Indique qu'il faut appliquer le style sur tout le doc 
% \makeatletter
% \let\ps@plain=\ps@fancy
% \makeatother

%opening
    \title{MONITOR YOUR INFRA}
\author{Nicolas Vadkerti Quentin Risdorfer}
\usepackage{listings} % Required for inserting code snippets
\usepackage[usenames,dvipsnames]{color} % Required for specifying custom colors and referring to colors by name

\definecolor{DarkGreen}{rgb}{0.0,0.4,0.0} % Comment color
\definecolor{highlight}{RGB}{255,251,204} % Code highlight color

\lstdefinestyle{Style1}{ % Define a style for your code snippet, multiple definitions can be made if, for example, you wish to insert multiple code snippets using different programming languages into one document
language=Bash, % Detects keywords, comments, strings, functions, etc for the language specified
backgroundcolor=\color{highlight}, % Set the background color for the snippet - useful for highlighting
basicstyle=\footnotesize\ttfamily, % The default font size and style of the code
breakatwhitespace=false, % If true, only allows line breaks at white space
breaklines=true, % Automatic line breaking (prevents code from protruding outside the box)
captionpos=b, % Sets the caption position: b for bottom; t for top
commentstyle=\usefont{T1}{pcr}{m}{sl}\color{DarkGreen}, % Style of comments within the code - dark green courier font
deletekeywords={}, % If you want to delete any keywords from the current language separate them by commas
%escapeinside={\%}, % This allows you to escape to LaTeX using the character in the bracket
firstnumber=1, % Line numbers begin at line 1
frame=single, % Frame around the code box, value can be: none, leftline, topline, bottomline, lines, single, shadowbox
frameround=tttt, % Rounds the corners of the frame for the top left, top right, bottom left and bottom right positions
keywordstyle=\color{Blue}\bf, % Functions are bold and blue
morekeywords={}, % Add any functions no included by default here separated by commas
numbers=left, % Location of line numbers, can take the values of: none, left, right
numbersep=10pt, % Distance of line numbers from the code box
numberstyle=\tiny\color{Gray}, % Style used for line numbers
rulecolor=\color{black}, % Frame border color
showstringspaces=false, % Don't put marks in string spaces
showtabs=false, % Display tabs in the code as lines
stepnumber=5, % The step distance between line numbers, i.e. how often will lines be numbered
stringstyle=\color{Purple}, % Strings are purple
tabsize=2
% literate={á}{{\'a}}1 {ã}{{\~a}}1 {é}{{\'e}}1,
% inputencoding=utf8
}

\newcommand{\insertcode}[2]{\begin{itemize}\item[]\lstinputlisting[caption=#2,label=#1,style=Style1]{#1}\end{itemize}} 


% \insertcode{"Scripts/example.pl"}{Nena would be proud.} 

\begin{document}

\maketitle


\url{https://github.com/SlaynPool/CR_MONITOR_YOUR_INFRA}



\section{Utilisation de SNMP comme vecteur de monitoring}
\subsection{Installez le client SNMP sous Linux}

\insertcode{commande/1.txt}{Installation d'un Client}

\insertcode{commande/2.txt}{Test d'interrogation}

Pour Autoriser les connections de l'exterieur, il faut :

\insertcode{commande/3.txt}{snmpd.conf}

\section{Utilisez le client SNMP afin de visualiser les informations des machines listées dans le "terrain de jeux"}
\subsection{Interrogation via SNMP du serveur ayant pour IP 10.6.0.1.}

\subsubsection{Dumper l’ensemble des informations du serveur distant via un snmpwalk}
\insertcode{commande/4.txt}{snmpwalk}

\subsubsection{Retrouver le système d’exploitation de la machine via un snmpget.}
\insertcode{commande/5.txt}{snmpget}

\subsubsection{Afficher l’arbre system de la mib à l’aide de la commande }
\insertcode{commande/6.txt}{Arbre de la mib SNMPv2}
\subsubsection{Traduisez en oid SNMPv2-MIB : :system et réciproquement}
\insertcode{commande/7.txt}{Traduction}

\section{Utilisation d’OMD comme logiciel de supervision SNMP}

Paquets à installer:

mk-check-agent: \url{store.iutbeziers.fr/check-mk-agent/}

OMD: \url{http://clusterfrak.com/sysops/app_installs/omd_install/}

\subsection{Supervisez avec OMD}



Créer notre site avec OMD:
\insertcode{commande/a.txt}{Création iutbeziers}

Premiers pas:
\insertcode{commande/b.txt}{Afficher status du site}


\insertcode{commande/c.txt}{Démarrer notre site IURBEZIERS}

\newpage
Modifier l'adresse IP du site:



  \begin{figure}[!h]
\centering
\includegraphics[scale=1]{image/omdip.png}
\caption{Modify IP}
\label{figure}

\end{figure}
\end{document}


